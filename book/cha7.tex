% LaTeX source for textbook ``How to think like a computer scientist''
% Copyright (C) 1999  Allen B. Downey



\chapter{Strings and things}
\label{strings}

\section{Different Strings available}

We have seen five types of values---booleans, characters, integers,
floating-point numbers and strings---but only four types of
variables---{\tt bool}, {\tt char}, {\tt int} and {\tt double}.  So
far we have no way to store a string in a variable or perform
operations on strings.

In fact, there are several kinds of variables in C++ that
can store strings.  One is a basic type that is part of the C++
language, sometimes called ``a native C string.''  The syntax
for C strings is a bit ugly, and using them requires some concepts
we have not covered yet, so for the most part we are going to
avoid them.
\section{C-Strings}
FIXME

The string type we are going to use is called {\tt string}, which is
one of the classes that belong to the C++ Standard Library.\footnote{You might be wondering what I mean by {\bf class}.It will be a few
more chapters before I can give a complete definition, but for now a
class is a collection of functions that defines the operations we
can perform on some type.  The {\tt string} class contains all
the functions that apply to {\tt string}s.}

Unfortunately, it is not possible to avoid C strings altogether.
In a few places in this chapter I will warn you about some problems
you might run into using {\tt string}s instead of C strings.

\section{{\tt string} variables}

You can create a variable with type {\tt string} in the usual
ways:

\begin{verbatim}
  string first;
  first = "Hello, ";
  string second = "world.";
\end{verbatim}
%
The first line creates a {\tt string} without giving it a value.
The second line assigns it the string value {\tt "Hello."}
The third line is a combined declaration and assignment, also
called an initialization.

Normally when string values like {\tt "Hello, "} or {\tt "world."}
appear, they are treated as C strings.  In this case, when we assign
them to a {\tt string} variable, they are converted automatically
to {\tt string} values.

We can output strings in the usual way:

\begin{verbatim}
  cout << first << second << endl;
\end{verbatim}
%

In order to compile this code, you will have to include the
header file for the {\tt string} class, and you will have
to add the file {\tt string} to the list of files you
want to compile.  The details of how to do this depend on your
programming environment.

Before proceeding, you should type in the program above and make
sure you can compile and run it.

\section{Extracting characters from a string}

Strings are called ``strings'' because they are made up of a sequence,
or string, of characters.  The first operation we are going to
perform on a string is to extract one of the characters.  C++
uses square brackets ({\tt [} and {\tt ]}) for this operation:

\begin{verbatim}
  string fruit = "banana";
  char letter = fruit[1];
  cout << letter << endl;
\end{verbatim}
%
The expression {\tt fruit[1]} indicates that I want character number 1
from the string named {\tt fruit}.  The result is stored in a {\tt
char} named {\tt letter}.  When I output the value of {\tt letter}, I
get a surprise:

\begin{verbatim}
a
\end{verbatim}
%
{\tt a} is not the first letter of {\tt "banana"}.  Unless you are a
computer scientist.  For perverse reasons, computer scientists always
start counting from zero.  The 0th letter (``zeroeth'') of {\tt
"banana"} is {\tt b}.  The 1th letter (``oneth'') is {\tt a} and the
2th (``twoeth'') letter is {\tt n}.

If you want the the zereoth letter of a string, you have to put
zero in the square brackets:

\begin{verbatim}
  char letter = fruit[0];
\end{verbatim}

\section{Length}
\index{string!length}
\index{length!string}

To find the length of a string (number of characters), we can
use the {\tt length} function.  The syntax for calling this
function is a little different from what we've seen before:

\begin{verbatim}
  int length;
  length = fruit.length();
\end{verbatim}
%
To describe this function call, we would say that we are {\bf
invoking} the length function on the string named {\tt fruit}.  This
vocabulary may seem strange, but we will see many more examples where
we invoke a function on an object.  The syntax for function invocation
is called ``dot notation,'' because the dot (period) separates the
name of the object, {\tt fruit}, from the name of the function, {\tt
length}.

{\tt length} takes no arguments, as indicated by the empty parentheses
{\tt ()}.  The return value is an integer, in this case 6.  Notice
that it is legal to have a variable with the same name as a function.

To find the last letter of a string, you might be tempted to
try something like

\begin{verbatim}
  int length = fruit.length();
  char last = fruit[length];       // WRONG!!
\end{verbatim}
%
That won't work.  The reason is that there is no 6th letter
in {\tt "banana"}.  Since we started counting at 0, the 6
letters are numbered from 0 to 5.  To get the last character,
you have to subtract 1 from {\tt length}.

\begin{verbatim}
  int length = fruit.length();
  char last = fruit[length-1];
\end{verbatim}

\section{Traversal}
\index{traverse}

A common thing to do with a string is
start at the beginning, select each character in turn, do
something to it, and continue until the end.  This pattern
of processing is called a {\bf traversal}.  A natural
way to encode a traversal is with a {\tt while} statement:

\begin{verbatim}
  int index = 0;
  while (index < fruit.length()) {
    char letter = fruit[index];
    cout << letter << endl;
    index = index + 1;
  }
\end{verbatim}
%
This loop traverses the string and outputs each letter on
a line by itself.  Notice that the condition is
{\tt index < fruit.length()}, which means that when
{\tt index} is equal to the length of the string, the
condition is false and the body of the loop is not executed.
The last character we access is the one with the
index {\tt fruit.length()-1}.

\index{loop variable}
\index{variable!loop}
\index{index}

The name of the loop variable is {\tt index}.  An {\bf
index} is a variable or value used to specify one member of an ordered
set, in this case the set of characters in the string.  The index
indicates (hence the name) which one you want.  The set has to be
ordered so that each letter has an index and each index
refers to a single character.

As an exercise, write a function that takes a {\tt string}
as an argument and that outputs the letters backwards, all on
one line.

\section{A run-time error}
\index{error!run-time}
\index{run-time error}

Way back in Section~\ref{run-time} I talked about run-time errors,
which are errors that don't appear until a program has started
running.

So far, you probably haven't seen many run-time errors, because we
haven't been doing many things that can cause one.  Well, now we are.
If you use the {\tt []} operator and you provide an index that is
negative or greater than {\tt length-1}, you will get a run-time
error and a message something like this:

\begin{verbatim}
index out of range: 6, string: banana
\end{verbatim}
%
Try it in your development environment and see how it looks.

\section{The {\tt find} function}
\index{find}

The {\tt string} class provides several other functions that you can
invoke on strings.  The {\tt find} function is like the opposite the
{\tt []} operator.  Instead of taking an index and extracting the
character at that index, {\tt find} takes a character and finds the
index where that character appears.

\begin{verbatim}
  string fruit = "banana";
  int index = fruit.find('a');
\end{verbatim}
%
This example finds the index of the letter {\tt 'a'} in the string.
In this case, the letter appears three times, so it is not obvious
what {\tt find} should do.  According to the documentation, it returns
the index of the {\em first} appearance, so the result is 1.  If the
given letter does not appear in the string, {\tt find} returns -1.

In addition, there is a
version of {\tt find} that takes another {\tt string} as
an argument and that finds the index where the substring
appears in the string.  For example,

\begin{verbatim}
  string fruit = "banana";
  int index = fruit.find("nan");
\end{verbatim}
%
This example returns the value 2.

You should remember from Section~\ref{overloading} that there
can be more than one function with the same name, as long as they
take a different number of parameters or different types.  In
this case, C++ knows which version of {\tt find} to invoke
by looking at the type of the argument we provide.

\section{Our own version of {\tt find}}

If we are looking for a letter in a {\tt string}, we may
not want to start at the beginning of the string.  One way
to generalize the {\tt find} function is to write a version
that takes an additional parameter---the index where we should
start looking.  Here is an implementation of this function.

\begin{verbatim}
int find (string s, char c, int i)
{
  while (i<s.length()) {
    if (s[i] == c) return i;
    i = i + 1;
  }
  return -1;
}
\end{verbatim}
%
Instead of invoking this function on a {\tt string}, like
the first version of {\tt find}, we have to pass the {\tt string}
as the first argument.  The other arguments are the character
we are looking for and the index where we should start.

\section{Looping and counting}
\label{loopcount}
\index{traverse!counting}
\index{loop!counting}

The following program counts the
number of times the letter {\tt 'a'} appears in a string:

\begin{verbatim}
  string fruit = "banana";
  int length = fruit.length();
  int count = 0;

  int index = 0;
  while (index < length) {
    if (fruit[index] == 'a') {
      count = count + 1;
    }
    index = index + 1;
  }
  cout << count << endl;
\end{verbatim}
%
This program demonstrates a common idiom, called a {\bf counter}.  The
variable {\tt count} is initialized to zero and then incremented each
time we find an {\tt 'a'}.  (To {\bf increment} is to increase by one;
it is the opposite of {\bf decrement}, and unrelated to {\bf
excrement}, which is a noun.)  When we exit the loop, {\tt count}
contains the result: the total number of a's.

\index{counter}
\index{increment}
\index{decrement}

As an exercise, encapsulate this code in a function named
{\tt countLetters}, and generalize it so that it accepts the
string and the letter as arguments.

\index{encapsulation}
\index{generalization}

As a second exercise, rewrite this function so that instead
of traversing the string, it uses the version of
{\tt find} we wrote in the previous section.


\section{String concatenation}

Interestingly, the {\tt +} operator can be used on strings;
it performs string {\bf concatenation}.  To concatenate means to
join the two operands end to end.  For example:

\begin{verbatim}
  string fruit = "banana";
  string bakedGood = " nut bread";
  string dessert = fruit + bakedGood;
  cout << dessert << endl;
\end{verbatim}
%
The output of this program is {\tt banana nut bread}.

Unfortunately, the {\tt +} operator does not work on native
C strings, so you cannot write something like

\begin{verbatim}
  string dessert = "banana" + " nut bread";
\end{verbatim}
%
because both operands are C strings.  As long as one of the
operands is  {\tt string}, though, C++ will automatically
convert the other.

It is also possible to concatenate a character onto the
beginning or end of a {\tt string}.  In the following example, we
will use concatenation and character arithmetic to output
an abecedarian series.

``Abecedarian'' refers to a series or list in which the elements
appear in alphabetical order.  For example, in Robert McCloskey's book
{\em Make Way for Ducklings}, the names of the ducklings are Jack,
Kack, Lack, Mack, Nack, Ouack, Pack and Quack.  Here is a loop that
outputs these names in order:

\begin{verbatim}
  string suffix = "ack";

  char letter = 'J';
  while (letter <= 'Q') {
    cout << letter + suffix << endl;
    letter++;
  }
\end{verbatim}
%
The output of this program is:

\begin{verbatim}
Jack
Kack
Lack
Mack
Nack
Oack
Pack
Qack
\end{verbatim}
%
Of course, that's not quite right because I've misspelled ``Ouack''
and ``Quack.''  As an exercise, modify the program to correct
this error.

Again, be careful to use string concatenation only with {\tt string}s
and not with native C strings.  Unfortunately, an expression like
{\tt letter + "ack"} is syntactically legal in C++, although it
produces a very strange result, at least in my development environment.

\section{{\tt string}s are mutable}
\index{class!string}
\index{immutable}
\index{string}

You can change the letters in a {\tt string} one at a time
using the {\tt []} operator on the left side of an assignment.
For example,

\begin{verbatim}
  string greeting = "Hello, world!";
  greeting[0] = 'J';
  cout << greeting << endl;
\end{verbatim}
%
produces the output {\tt Jello, world!}.


\section{{\tt string}s are comparable}
\label{incomparable}
\index{class!string}
\index{comparison!string}
\index{string}

All the comparison operators that work on {\tt int}s and
{\tt double}s also work on {\tt strings}.  For example,
if you want to know if two strings are equal:

\begin{verbatim}
  if (word == "banana") {
    cout << "Yes, we have no bananas!" << endl;
  }
\end{verbatim}
%
The other comparison operations are useful for putting words
in alphabetical order.

\begin{verbatim}
  if (word < "banana") {
    cout << "Your word, " << word << ", comes before banana." << endl;
  } else if (word > "banana") {
    cout << "Your word, " << word << ", comes after banana." << endl;
  } else {
    cout << "Yes, we have no bananas!" << endl;
  }
\end{verbatim}
%
You should be aware, though, that the {\tt string} class does
not handle upper and lower case letters the same way that people
do.  All the upper case letters come before all the lower case
letters.  As a result,

\begin{verbatim}
Your word, Zebra, comes before banana.
\end{verbatim}
%
A common way to address this problem is to convert strings to a
standard format, like all lower-case, before performing the
comparison.  The next sections explains how.  I will not address the
more difficult problem, which is making the program realize that
zebras are not fruit.

\section{Character classification}

It is often useful to examine a character and test whether
it is upper or lower case, or whether it is a character or
a digit.  C++ provides a library of functions that perform
this kind of character classification.  In order to use these
functions, you have to include the header file {\tt ctype.h}.

\begin{verbatim}
  char letter = 'a';
  if (isalpha(letter)) {
    cout << "The character " << letter << " is a letter." << endl;
  }
\end{verbatim}
%
You might expect the return value from {\tt isalpha} to
be a {\tt bool}, but for reasons I don't even want to think
about, it is actually an integer that is
0 if the argument is not a letter, and some non-zero value
if it is.

This oddity is not as inconvenient as it seems, because it is
legal to use this kind of integer in a conditional, as shown
in the example.  The value 0 is treated as {\tt false}, and
all non-zero values are treated as {\tt true}.

Technically, this sort of thing should not be allowed---integers are
not the same thing as boolean values.  Nevertheless, the C++ habit of
converting automatically between types can be useful.

Other character classification functions include {\tt isdigit}, which
identifies the digits 0 through 9, and {\tt isspace}, which identifies
all kinds of ``white'' space, including spaces, tabs, newlines, and a
few others.  There are also {\tt isupper} and {\tt islower}, which
distinguish upper and lower case letters.

Finally, there are two functions that convert letters from one
case to the other, called {\tt toupper} and {\tt tolower}.  Both take
a single character as a parameter and return a (possibly
converted) character.

\begin{verbatim}
  char letter = 'a';
  letter = toupper (letter);
  cout << letter << endl;
\end{verbatim}
%
The output of this code is {\tt A}.

As an exercise, use the character classification and conversion
library to write functions named {\tt stringToUpper} and
{\tt stringToLower} that take a single {\tt string} as
a parameter, and that modify the string by converting all the
letters to upper or lower case.  The return type should be
{\tt void}.

\section{Other {\tt string} functions}

This chapter does not cover all the {\tt string} functions.
Two additional ones, {\tt c\_str} and {\tt substr}, are covered
in Section~\ref{finput} and Section~\ref{parsing}.

\section{Passing to functions}
FIXME: Prefer std::string\_view function parameters (over const std::string\&) 
\section{Glossary}

\begin{description}

\item[object:] A collection of related data that comes with a set of
functions that operate on it.  The objects we have used so far are the
{\tt cout} object provided by the system, and {\tt string}s.

\item[index:]  A variable or value used to select one of the
members of an ordered set, like a character from a string.

\item[traverse:]  To iterate through all the elements of a set
performing a similar operation on each.

\item[counter:]  A variable used to count something, usually
initialized to zero and then incremented.

\item[increment:]  Increase the value of a variable by one.
The increment operator in C++ is {\tt ++}.  In fact, that's
why C++ is called C++, because it is meant to be one better
than C.

\item[decrement:]  Decrease the value of a variable by one.
The decrement operator in C++ is {\tt --}.

\item[concatenate:] To join two operands end-to-end.

\index{object}
\index{index}
\index{traverse}
\index{counter}
\index{increment}
\index{decrement}
\index{concatenate}

\end{description}
