% LaTeX source for textbook ``ThinkCPP , a game perspective''
% Copyright (C) 2023  Lisa Patacchiola and Allen B Downey

\chapter{Iteration}


\section{Multiple assignment}
\index{assignment}
\index{statement!assignment}
\index{multiple assignment}

I haven't said much about it, but it is legal in C++ to
make more than one assignment to the same variable.  The
effect of the second assignment is to replace the old value
of the variable with a new value.

\begin{verbatim}
  int fred = 5;
  cout << fred;
  fred = 7;
  cout << fred;
\end{verbatim}
%
The output of this program is {\tt 57}, because the first
time we print {\tt fred} his value is 5, and the second time
his value is 7.

This kind of {\bf multiple assignment} is the reason I
described variables as a {\em container} for values.  When
you assign a value to a variable, you change the contents of
the container, as shown in the figure:

\vspace{0.1in}
\centerline{\epsfig{figure=assign2.eps}}
\vspace{0.1in}

When there are multiple assignments to a variable, it is especially
important to distinguish between an assignment statement and a
statement of equality.  Because C++ uses the {\tt =} symbol for
assignment, it is tempting to interpret a statement like {\tt a = b}
as a statement of equality.  It is not!

First of all, equality is commutative, and assignment is not.
For example, in mathematics if $a = 7$ then $7 = a$.  But in
C++ the statement {\tt a = 7;} is legal, and {\tt 7 = a;}
is not.

Furthermore, in mathematics, a statement of equality is true
for all time.  If $a = b$ now, then $a$ will always equal $b$.
In C++, an assignment statement can make two variables equal,
but they don't have to stay that way!

\begin{verbatim}
  int a = 5;
  int b = a;     // a and b are now equal
  a = 3;         // a and b are no longer equal
\end{verbatim}
%
The third line changes the value of {\tt a} but it does not
change the value of {\tt b}, and so they are no longer equal.
In many programming languages an alternate symbol is used
for assignment, such as {\tt <-} or {\tt :=}, in order to
avoid confusion.

Although multiple assignment is frequently useful, you should
use it with caution.  If the values of variables are changing
constantly in different parts of the program, it can make
the code difficult to read and debug.

\section{Iteration}
\index{iteration}

One of the things computers are often used for is the automation
of repetitive tasks.  Repeating identical or similar tasks without
making errors is something that computers do well and people do
poorly.

FIXME - remove recusion info from here
We have seen programs that use recursion to perform
repetition, such as {\tt nLines} and {\tt countdown}.  This
type of repetition is called {\bf iteration}, and C++ provides
several language features that make it easier to write iterative
programs.

FIXME
The two features we are going to look at are the {\tt while}
statement and the {\tt for} statement.

\section{The {\tt while} statement}
\index{statement!while}
\index{while statement}

Using a {\tt while} statement, we can rewrite {\tt countdown}:

\begin{verbatim}
int countdown (int n) {
  while (n > 0) {
    cout << n << endl;
    n = n-1;
  }
  cout << "Blastoff!" << endl;
  return 0;
}
\end{verbatim}
%
You can almost read a {\tt while} statement as if it were
English.  What this means is, ``While {\tt n} is greater than
zero, continue displaying the value of {\tt n} and then reducing
the value of {\tt n} by 1.  When you get to zero, output the
word `Blastoff!'''

More formally, the flow of execution for a {\tt while} statement
is as follows:

\begin{enumerate}

\item Evaluate the condition in parentheses, yielding {\tt true}
or {\tt false}.

\item If the condition is false, exit the {\tt while} statement
and continue execution at the next statement.

\item If the condition is true, execute each of the statements
between the curly-braces, and then go back to step 1.

\end{enumerate}

This type of flow is called a {\bf loop} because the third step loops
back around to the top.  Notice that if the condition is false the
first time through the loop, the statements inside the loop are
never executed.  The statements inside the loop are called
the {\bf body} of the loop.

Here is a while loop simulator. \url{https://lpatacch.github.io/thinkCPPGamesEx/WhilePractice.html}. 
With this, you can change the 
initial value, the condition it is checking (\textless, \textgreater, ==, !=, \textgreater=, \textless=), the value checked,
and how the variable will change. Feel free to use this
to understand how while loops work.
\index{loop}
\index{loop!body}
\index{loop!infinite}
\index{body!loop}
\index{infinite loop}

The body of the loop should change the value of
one or more variables so that, eventually, the condition becomes
false and the loop terminates.  Otherwise the loop will repeat
forever, which is called an {\bf infinite loop}.  An endless
source of amusement for computer scientists is the observation
that the directions on shampoo, ``Lather, rinse, repeat,'' are
an infinite loop.

In the case of {\tt countdown}, we can prove that the loop
will terminate because we know that the value of {\tt n} is
finite, and we can see that the value of {\tt n} gets smaller
each time through the loop (each {\bf iteration}), so
eventually we have to get to zero.  In other cases it is not
so easy to tell:

\begin{verbatim}
  void sequence (int n) {
    while (n != 1) {
      cout << n << endl;
      if (n%2 == 0) {           // n is even
        n = n / 2;
      } else {                  // n is odd
        n = n*3 + 1;
      }
    }
  }
\end{verbatim}
%
The condition for this loop is {\tt n != 1}, so the loop
will continue until {\tt n} is 1, which will make the condition
false.

At each iteration, the program outputs the value of {\tt n} and then
checks whether it is even or odd.  If it is even, the value of
{\tt n} is divided by two.  If it is odd, the value is replaced
by $3n+1$.  For example, if the starting value (the argument passed
to {\tt sequence}) is 3, the resulting sequence is
3, 10, 5, 16, 8, 4, 2, 1.

Since {\tt n} sometimes increases and sometimes decreases, there is no
obvious proof that {\tt n} will ever reach 1, or that the program will
terminate.  For some particular values of {\tt n}, we can prove
termination.  For example, if the starting value is a power of two, then
the value of {\tt n} will be even every time through the loop, until
we get to 1.  The previous example ends with such a sequence,
starting with 16.

Particular values aside, the interesting question is whether
we can prove that this program terminates for {\em all} values of n.
So far, no one has been able to prove it {\em or} disprove it!

\section{Definite loop}
FIXME count example
\section{Indefinite loop}
FIXME sentinel example
\section{New Operators}
FIXME
Things like the += operator
\section{Increment and decrement operators}
\index{operator!increment}
\index{operator!decrement}

Incrementing and decrementing are such common operations that C++
provides special operators for them.  The {\tt ++} operator adds one
to the current value of an {\tt int}, {\tt char} or {\tt double}, and
{\tt --} subtracts one.  Neither operator works on {\tt string}s,
and neither {\em should} be used on {\tt bool}s.

Technically, it is legal to increment a variable and use it
in an expression at the same time.  For example, you might see
something like:

\begin{verbatim}
  cout << i++ << endl;
\end{verbatim}
%
Looking at this, it is not clear whether the increment will
take effect before or after the value is displayed.  Because
expressions like this tend to be confusing, I would discourage
you from using them.  In fact, to discourage you even more,
I'm not going to tell you what the result is.  If you really
want to know, you can try it.

Using the increment operators, we can rewrite the letter-counter:

FIXME - change example. no arrays yet.
\begin{verbatim}
  int index = 0;
  while (index < length) {
    if (fruit[index] == 'a') {
      count++;
    }
    index++;
  }
\end{verbatim}
%
It is a common error to write something like

\begin{verbatim}
  index = index++;             // WRONG!!
\end{verbatim}
%
Unfortunately, this is syntactically legal, so the compiler
will not warn you.  The effect of this statement is to leave
the value of {\tt index} unchanged.  This is often a difficult
bug to track down.

Remember, you can write {\tt index = index +1;}, or you
can write {\tt index++;}, but you shouldn't mix them.

\section{Tables}
\index{table}
\index{logarithm}

One of the things loops are good for is generating
tabular data.  For example, before computers were readily available,
people had to calculate logarithms, sines and cosines, and other
common mathematical functions by hand.
To make that easier, there were books containing long tables
where you could find the values of various functions.
Creating these tables was slow and boring, and the result
tended to be full of errors.

When computers appeared on the scene, one of the initial reactions
was, ``This is great!  We can use the computers to generate the
tables, so there will be no errors.''  That turned out to be true
(mostly), but shortsighted.  Soon thereafter computers and
calculators were so pervasive that the tables became obsolete.

Well, almost.  It turns out that for some operations, computers
use tables of values to get an approximate answer, and then
perform computations to improve the approximation.  In some
cases, there have been errors in the underlying tables, most
famously in the table the original Intel Pentium used to perform
floating-point division.

\index{division!floating-point}

Although a ``log table'' is not as useful as it once was, it still
makes a good example of iteration.  The following program outputs a
sequence of values in the left column and their logarithms in the
right column:

\begin{verbatim}
  double x = 1.0;
  while (x < 10.0) {
    cout << x << "\t" << log(x) << "\n";
    x = x + 1.0;
  }
\end{verbatim}
%
The sequence \verb+\t+ represents a {\bf tab} character.
The
sequence \verb+\n+ represents a newline character.  These sequences
can be included anywhere in a string, although in these examples
the sequence is the whole string.

A tab character causes the cursor to shift to the right until
it reaches one of the {\bf tab stops}, which are normally every
eight characters.  As we will see in a minute, tabs are useful
for making columns of text line up.

A newline character has exactly the same effect as {\tt endl};
it causes the cursor to move on to the next line.  Usually if
a newline character appears by itself, I use {\tt endl}, but
if it appears as part of a string, I use \verb+\n+.

The output of this program is

\begin{verbatim}
1      0
2      0.693147
3      1.09861
4      1.38629
5      1.60944
6      1.79176
7      1.94591
8      2.07944
9      2.19722
\end{verbatim}
%
If these values seem odd, remember that the {\tt log} function uses
base $e$.  Since powers of two are so important in computer science,
we often want to find logarithms with respect to base 2.  To do that,
we can use the following formula:

\[ \log_2 x = \frac {log_e x}{log_e 2} \]
%
Changing the output statement to

\begin{verbatim}
      cout << x << "\t" << log(x) / log(2.0) << endl;
\end{verbatim}
%
yields

\begin{verbatim}
1      0
2      1
3      1.58496
4      2
5      2.32193
6      2.58496
7      2.80735
8      3
9      3.16993
\end{verbatim}
%
We can see that 1, 2, 4 and 8 are powers of two, because
their logarithms base 2 are round numbers.  If we wanted to find
the logarithms of other powers of two, we could modify the
program like this:

\begin{verbatim}
  double x = 1.0;
  while (x < 100.0) {
    cout << x << "\t" << log(x) / log(2.0) << endl;
    x = x * 2.0;
  }
\end{verbatim}
%
Now instead of adding something to {\tt x} each time through
the loop, which yields an arithmetic sequence, we multiply
{\tt x} by something, yielding a {\bf geometric} sequence.
The result is:

\begin{verbatim}
1      0
2      1
4      2
8      3
16     4
32     5
64     6
\end{verbatim}
%
Because we are using tab characters between the columns, the
position of the second column does not depend on the number
of digits in the first column.

Log tables may not be useful any more, but for computer scientists,
knowing the powers of two is!  As an exercise, modify this program
so that it outputs the powers of two up to 65536
(that's $2^{16}$).  Print it out and memorize it.

\section{Two-dimensional tables}
\index{table!two-dimensional}

A two-dimensional table is a table where you choose a row and
a column and read the value at the intersection.  A multiplication
table is a good example.  Let's say you wanted to print a
multiplication table for the values from 1 to 6.

A good way to start is to write a simple loop that prints
the multiples of 2, all on one line.

\begin{verbatim}
  int i = 1;
  while (i <= 6) {
    cout << 2*i << "   ";
    i = i + 1;
  }
  cout << endl;
\end{verbatim}
%
The first line initializes a variable named {\tt i}, which is
going to act as a counter, or {\bf loop variable}.  As the
loop executes, the value of {\tt i} increases from 1 to 6,
and then when {\tt i} is 7, the loop terminates.  Each
time through the loop, we print the value {\tt 2*i} followed
by three spaces.  By omitting the {\tt endl} from the
first output statement, we get 
all the output on a single line.

\index{loop variable}
\index{variable!loop}

The output of this program is:

\begin{verbatim}
2   4   6   8   10   12
\end{verbatim}
%
So far, so good.  The next step is to {\bf encapsulate} and {\bf
generalize}.

\section{{\tt do-while} loop}
FIXME

\section{{\tt for} loops}
FIXME

The loops we have written so far have a number of elements
in common.  All of them start by initializing a variable;
they have a test, or condition, that depends on that variable;
and inside the loop they do something to that variable,
like increment it.

\index{loop!for}
\index{for}
\index{statement!for}

This type of loop is so common that there is an alternate
loop statement, called {\tt for}, that expresses it more
concisely.  The general syntax looks like this:

\begin{verbatim}
  for (INITIALIZER; CONDITION; INCREMENTOR) {
    BODY
  }
\end{verbatim}
%
This statement is exactly equivalent to

\begin{verbatim}
  INITIALIZER;
  while (CONDITION) {
    BODY
    INCREMENTOR
  }
\end{verbatim}
%
except that it is more concise and, since it puts all the
loop-related statements in one place, it is easier to read.
For example:

\begin{verbatim}
  int i;
  for (i = 0; i < 4; i++) {
    cout << count[i] << endl;
  }
\end{verbatim}
%
is equivalent to 

\begin{verbatim}
  int i = 0;
  while (i < 4) {
    cout << count[i] << endl;
    i++;
  }
\end{verbatim}

FIXME - example of for loop with a decrement instead
FIXME - example of for loop with a step of 5

\section{Break and Continue}
FIXME add examples of how this works
\section{Glossary}

\begin{description}

\item[loop:]  A statement that executes repeatedly while a
condition is true or until some condition is satisfied.

\item[infinite loop:]  A loop whose condition is always true.

\item[body:]  The statements inside the loop.

\item[iteration:]  One pass through (execution of) the body
of the loop, including the evaluation of the condition.

\item[tab:] A special character, written as \verb+\t+ in C++,
that causes the cursor to move to the next tab stop on the
current line.

\item[encapsulate:]  To divide a large complex program into
components (like functions) and isolate the components from
each other (for example, by using local variables).

\item[local variable:]  A variable that is declared inside
a function and that exists only within that function.  Local variables
cannot be accessed from outside their home function, and do not
interfere with any other functions.

\item[generalize:]  To replace something unnecessarily specific
(like a constant value) with something appropriately general
(like a variable or parameter).  Generalization makes code more
versatile, more likely to be reused, and sometimes even easier
to write.

\item[development plan:]  A process for developing a program.
In this chapter, I demonstrated a style of development based on
developing code to do simple, specific things, and then encapsulating
and generalizing.

\index{loop}
\index{infinite loop}
\index{body}
\index{tab}
\index{loop!infinite}
\index{iteration}
\index{encapsulation}
\index{generalization}
\index{local variable}
\index{variable!local}
\index{program development}

\end{description}

