% LaTeX source for textbook ``Think C++, a game perspective''
% Copyright (C) 2023  Lisa Patacchiola

\chapter{Preface}

\section{Notes from Lisa}
Hi! If you are reading this note, you are reading an alpha version of this book. That means that I have not completely converted the book over the the format I eventually want to have. Double check my github account that you have the most recent version.

If this is the most recent version, here is the information so far.

Chapter 1 and 2 has some work done. I have changed some of the examples to have more of a game flavor. I have moved Functions a bit further back in the book and have moved If statements up to Chapter 3 and loops have been moved up to Chapter 4. I have also brought switch statements and for loops up as well. Chapter 3 has been started, but has not been converted over yet. You will still find "FIXME"s in the chapter, and I have not switched the examples to something more game-like. I have put stubs for flowcharts and pseudo code, but I have not put anything in there yet. Chapter 4 has been started as well. There have been some flowcharts added. Other chapters have had small changes, mostly so the previous links are no longer broken.

I have added an appendix to talk about how to turn on warnings on Replit and how to setup Visual Studio Code on a Windows PC. There is also a new chapter (17) on advanced features like lambda expressions. 


More notes:

I was previously using a different book for my classes. It did have lots of information, but it seemed like it had too much information. Students couldn't get through the reading because they found it too long and complex.

I started to use ThinkCPP, but I noticed that the book was using various coding practices that were no longer acceptable. (They were perfect when it was published, but C++ has changed a bit since then.) I first was only going to change a few examples, but after a few student questions, I decided to do a larger change to this book.


\section{Acknowledgments}

\section{Contributions}
The first edition was called "Think like a computer scientist" and was written by Allen B. Downey. 

Lisa Patacchiola wrote the new additions and created Replit projects for the examples. 


