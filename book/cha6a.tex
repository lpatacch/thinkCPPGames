We have seen many types of variables so far. We have seen integers, characters and booleans. The next type of variable
is a bit different than the others. It allows you to have
many variables of the same type grouped together under the
same variable name. This type of variable is called an array.
\section{Traditional array}
\index{array}
\index{array!traditional}
First, we will be looking at the type of array that came from
the language "C". Traditional arrays can be found in some legacy programs, or code that has to interface with modules written in C.

This type is defined with square brackets ([]). The definition explains what type of array it is, and that it
is an array. For example, here is an array of integers:

\begin{verbatim}
    int weights[5];
\end{verbatim}
The above code created an array of type integer. It can hold
5 values, and the variable is named "weights". When the array is created, it does not set values in the array, so the numbers could be anything. To initialize the values to 0, 
you can do the following:
\begin{verbatim}
    int weights[5]{};
\end{verbatim}
That would set all 5 slots to 0. If you have certain values that you want to be initialized,
you can do something similar:
\begin{verbatim}
    int weights[5]{1,2,3,4,5};
\end{verbatim}
That will set the first element to 1, the second to 2 and so on. The great thing about this method is that it catches if you
put too many numbers:
\begin{verbatim}
    int weights[5]{1,2,3,4,5,6}; // WRONG!
\end{verbatim}
This will cause an error and not compile because this code
is attempting to place 6 items in 5 slots.

Next step is how to get to the values, and change the values.
These 5 values are indexed. You can access the variables by
using an index number. Let's say I want to know what the first weight is. I can print it out like:

\begin{verbatim}
    std::cout << "Weight 1 is "<< weight[0];
\end{verbatim}
You may have noticed that this first index is 0. When computers index arrays, the first item is at zero. Also,
the second is at 1 index, and the last item is at index
size-1. For example, in the weights array, which has a size
of 5, the last item is at index 4. I can print it like this:
\begin{verbatim}
    std::cout << "Weight 5 is "<< weight[4];
\end{verbatim}
If I want to change the value, I can use the index. Here
is an example of me setting the first element of weights
to 15.
\begin{verbatim}
    weight[0] = 15;
    std::cout << "Weight 1 is "<< weight[0];
\end{verbatim}

It is possible to send this type of variable into a
function, but I am not going to cover how to do that
in this book. (When a traditional array is sent to a function, it loses information about it's size and is
converted into a pointer.) Instead, I suggest you use some of the
other helper functions to convert an array to and from
one of the other types available. 


Although this type of variable does work, it
can be hard to understand how to use. If you are not careful, 
it is possible to access memory beyond where you are supposed to and could cause a memory corruption
The \href{https://isocpp.github.io/CppCoreGuidelines/CppCoreGuidelines#p7-catch-run-time-errors-early}{Guidelines} show many ways that this can cause a problem. . For this reason, Modern C++ suggests avoiding the traditional array if you can and use some of the 
newer array types\footnote{As of this books writing, the current standard is to avoid using the "pointer" types. So,items to avoid -- traditional arrays, c-strings, and pointers.}. We will cover the first one now.

\section{std::array}
\index{std::array}
\label{stdarray}
\index{array!std}
In C++11, {\tt std::array} was made to have an array that works like
some of the other C++ containers do. They keep the information about 
their size when sent into a function, and
copying code works the way that you would expect. There is
also some safety code that checks to make sure you are not 
accessing memory that you shouldn't be. std::array is a templated class
that is part of the C++ Standard Library. \footnote{Templates and classes
are both keywords we didn't talk about yet. I will be talking about both
later on in the book. For now, know that templates are a way that you can
get different types to work with the same code. Classes are a way to organize
functions and data together.}
In order to use this
type of array, you need to \#include<array>. That command can
be up the top of your file, with your other include statements. Here is an example of making a similar weights array, but with std::array instead:

\begin{lstlisting}
  std::array<int, 5> weight; //create an array of size 5
\end{lstlisting}
This code used the less than and greater than as part of
the definition. The first thing that is listed is the type 
of std::array we are making. This is making an integer array.
There is a comma, then we list the number of items in the 
std::array. The name of the variable is listed after the greater than sign. We can initialize in similar ways to
the traditional array:
\begin{lstlisting}
  std::array<int, 5> weight{}; //This sets all the values to 0
  // weights2 has index 1 to be 0 and so on.
  std::array<int, 5> weight2{1, 2, 3, 4, 5}; 
\end{lstlisting}
Note, in some older versions of the compiler (older than C++14), you would need the following syntax:
\begin{lstlisting}
  // weights2 has index 0 to be 1 and so on.
  std::array<int, 5> weight2{ {1, 2, 3, 4, 5} }; 
\end{lstlisting}

Also, if you are using C++17 or above, std::array can
figure out the size and type from the initialization. So
this could work instead:
\begin{lstlisting}
  // C++17 and later can figure out the size and type
  // by the values that are initialized
  std::array weight3{ 1, 2, 3, 4, 5 }; 
\end{lstlisting}

The code that we used for the traditional arrays to get
or set values will work for std::arrays too. But, there
are even better calls that checks the index and creates
an error (called an exception) if we try to access something
beyond where we should. This code uses a function named "at".
\begin{lstlisting}
  std::array<int, 5> weight; //create an array of size 5

  // The new commands to set and view
  weight.at(0) = 7;  // sets the 0 index to 7
  std::cout << weight.at(0); //prints the 0 index
  
  std::cout << weight.at(6); // WRONG -- Too far! 
\end{lstlisting}
Each time you use the "at" command, it checks to make sure the index you are using is valid. If it is, the code allows you to see or set what is at the index. If it is an invalid index, it will cause an error. In the above code, trying to view the value at the seventh item would cause an
error. This is another run time error. 

\subsection{Conversions to and from traditional arrays}
There are occasional times where you will need to deal with traditional arrays.
One way to handle this is to convert the traditional array to a std::array.
You could copy each item with a for loop, but there are is an even easier method.

\subsubsection{std::to\_array}
\index{to\_array}
std::to\_array was added to the language in C++20. By using this command, you
can use a traditional array to initialize a std::array. After it is created, you 
can use the safer array for the rest of your code:
\begin{lstlisting}
  // create a traditional array
  int gems[]{75, 67, 50};
  
  //convert it over to a std::array
  std::array gemarray{std::to_array(gems)};
  
  // look! It worked!
  std::cout << gemarray.size();  
\end{lstlisting}
\subsubsection{.data()}
If you need to use a C library that expects a parameter to be a traditional array,
you can get the pointer to the std::array data with the data() method. Here is an
example:
\begin{lstlisting}
    // This will send an array, and the array size to the 
    // function that requires a traditional array
    pointerFunction(gemarray.data(), gemarray.size());
\end{lstlisting}
\section{A new way to send an array into a function}
C++20 added a different way to send an array into a function. This particular
method works for traditional arrays, std::arrays, and vectors. It is called
a span. It is also a templated class.
\index{spans}
A span does not own the memory, it just has access to it. That means that
it can not delete the memory, but it can read and write to it. So, if
you are using code that has a traditional array, you can use the array
through span. It will not create more memory, just allow you to use it
differently. Here is an example of code that takes some type of int array
and calculates the average of what was sent.
\begin{lstlisting}
int average(std::span<int> items){
  int avg;
  int total{0};
  // if there is nothing in the span, the avg is 0
  if (items.size() == 0)
  {
    return 0;
  }
  // loop through the items, add them together
  for (const auto &item:items){
      total += item;
    }
    // calculate the average
  int valsize = static_cast< int >(items.size());
  avg = total/valsize;
  return avg;
}
\end{lstlisting}
The average code above can be used with a traditional array, std::array or even a 
std::vector. The following code shows how to call average.
\begin{lstlisting}
  std::array weight{1, 2, 3, 4, 5}; // a std::array
  int gems[]{75, 67, 50}; // a traditional array
 
  std::cout << "Avg Gems " << average(gems) << std::endl;
  std::cout << "Avg weight "<< average(weight) << std::endl;    
\end{lstlisting}
Note: this only works with traditional arrays if it hasn't decayed into a pointer.
(This can happen if you send it into another function.) If that happened, you will 
need to create a span with the size information.
\begin{lstlisting}
    average(std::span{pointerArray,10});
\end{lstlisting}
Spans have a lot more features. You can select just part of a span to create
another smaller span, and much more. But, we will not be covering that material
right now.
\section{And wait, there is more!}
The new array, std::array solved many problems, but it
still has some of the drawbacks. A std::array has a certain 
size, and there is no way to change it while it is running. There is an even better version of arrays that is flexible about it's size called "Vectors". We will cover vectors in Chapter~\ref{vectors}.