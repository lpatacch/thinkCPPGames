% LaTeX source for textbook ``ThinkCPP , a game perspective''
% Copyright (C) 2023  Lisa Patacchiola

\chapter{Advanced Topics}

\section{Function Pointer}
Just like you have pointers to variables, you can even have pointers to functions. (C++ uses this behind the scenes to implement object member and virtual functions.)

Function pointers are pointers that instead of pointing to data, point to a certain location in the executable code. Initializing a function pointer looks very similar to creating a function:

int (*match)(int key1, int key2);

The big difference is the extra * before the function name and that the function name and the star are surrounded by parentheses. 

This definition is a little tricky to understand. It gets even harder to understand when you have a function pointer returned from another function.
\begin{verbatim}
int (*HowToMatch(const int type)) (int, int) { 
 if (type == 1) return &match1; 
    else return &match2; 
} 
\end{verbatim}

Most people use a typedef when they create a function pointer to make it look more like a variable. For example, you can create match like:

\begin{verbatim}
typedef int (*Match)(int,int);

Match match;    
\end{verbatim}

 And, the function that returns a function pointer would just be:
 \begin{verbatim}
Match HowToMatch(const int type) { 
 if (type = = 1) return &match1; 
    else return &match2; 
} 
\end{verbatim}

You may have noticed in the earlier code, to initialize the function pointer to a function, just use =. For example, assuming match\_int is a function that will be later in the program.

\begin{verbatim}
int match_int(int key1, int key2);

match = match_int;    
or
match = &match_int;
\end{verbatim}


To call a function using a function pointer, dereference the pointer using the asterisk symbol or simply use the pointer variable name as if it were the function name:

\begin{verbatim}
retval = (*match)(x, y);

or

retval = match(x, y);    
\end{verbatim}

 
Pointers to functions can be passed to functions, returned from functions, stored in arrays, and assigned to other function pointers. They are often used for callback functions and in algorithms. They are quite powerful, but they do have some problems. The definition of the pointers can be confusing \footnote{std::function, described in Section ~\ref{stdfunction}, was created in C++11 to avoid the need to use typedef.} The functions can not keep any state information, so a new function would have to be created for each individual use. Each function would be taking up a namespace. Also, the compiler will not know what function will be set inside the pointer, so will not be able to call the function inline. 

For all those reasons, more recent versions of C++ have improved methods for the same or similar behavior. 

\section{Function Objects}
\index{function objects}
\index{functor}
Function Objects, also called functors, are a way to create a function via a class. This subject was introduced when we introduced overloading operators in \ref{functionobjects}. This particular object uses the overloaded parenthesis operator in order to allow the function object to be "called".

\begin{lstlisting}
//this creates the function object
class SayYayNumPlus {
  int num;  // This allows the function to have a "state"
public:
  SayYayNumPlus(int t):num(t){}
  void operator()(int i) const { 
        std::cout << "Yay "<< i+num << '\n'; 
        }
};

// The following would be in main
  SayYayNumPlus plus25(25);  // create the function object
                             // and put 25 in the "num"
  plus25(10);   // Output "Yay 35"
\end{lstlisting}
The above example shows the power of this object.
The constructor allows this class to keep a value that can
be used later. Then, in main, the object was created
and then can later be used. 

These objects can be passed into functions like a function
pointer. Unlike function pointers, the compiler know what
function it will call and can inline the function call.
\section{Lambda Expressions}
\index{lambda expressions}
Lambda expressions, also known as Lambda functions, and lambda were introduced to C++ in C++11.
This feature allows for inline unnamed functions that can be passed as parameters to other functions, or called where it was defined. Lambdas tend to be used when sending code to algorithms (like a sorting algorithm), or to asynchronous functions (code that will run at a different time.)

Here is an the format that many lambda expressions take:
\begin{verbatim}
 [capture](parameters){
    function body
 }
\end{verbatim}

Here is an example of a lambda expression being used to discover the smallest item in a range.\footnote{Normally you would not need this lambda expression because the default compare (using \textless) or the standard library compare function object would have been fine with this example. Lambda expressions or functors are usually needed when the default comparison would not work.} The code takes the vector, named v, and finds the smallest
item in the vector. It uses the expression in order to compare each item. Min\_Element
takes each element in turn and passes it as a parameter to the lambda expression. The
expression returns the result of the comparison.

\begin{lstlisting}

  std::vector<int> v{1, 7, 3, 4, 0};

  auto l = std::min_element(v.begin(), v.end(), 
              [](int x, int y) { return x < y; }); // lambda
  std::cout << "Min is " << *l << "\n";
  
\end{lstlisting}

\subsection{Generic Lambda}
In C++14, the above expression can be changed to a generic lambda by using "auto" for the parameter types. This will allow this call to be used by many different types, not just int. \footnote{Internally, this turns this code to an unnamed, templated functor.}:

\begin{lstlisting}
 [](auto x, auto y) { return x < y; }    
\end{lstlisting}

\subsection{Constant expressions}
\index{constexpr}
C++17 allows lambda expressions to be constexpr if it can be evaluated at compile time. If the compiler discovers that it can, the lambda will automatically be changed to constexpr. But, you can declare it explicitly.

\begin{verbatim}
 []() constexpr {
     function body
 }    
\end{verbatim}
For example, if there is something that will always return the same value:
\begin{lstlisting}
auto w = []() constexpr { return 7;}
\end{lstlisting}
This should automatically be set to a constexpr, but it will not hurt for you
to declare it anyway.


You may have noticed in some of the above examples that you can set a variable to a lambda expression. 

\begin{lstlisting}
auto var = [] () { 
    std::cout << "Hello, I'm in a lambda expression"; 
    };
    
var(); // This is how you call it
\end{lstlisting}

\subsection{Return Values}
In simple cases, the return value of a lambda expression can be deduced by the compiler so it does not need to be specified. But, more complex code needs to be explicitly listed. That is possible by using " -\textgreater ". This would be the format with the return type added.

\begin{verbatim}
 [capture](parameters)->returntype {
    function body
 }    
\end{verbatim}
Below is an example when you would need to set the return type explicitly. If you notice,
one return could be returning an integer, and another could be returning a double. The return value needs to be set explicitly because the compiler might not be able to guess what was expected:

\begin{lstlisting}
auto whattodo = []  (int num1, int num2,  
        string whichmath) -> double {

  if (whichmath == "avg") {
    // returns double here
    return (num1 + num2) / 2.0;
  } 
  else {
    // returns int here
    return num1 + num2;
  }
};
\end{lstlisting}

\subsection{Captures}
Normally, the code in the lambda body can not access variables from the
functions that are around it. If the body needs to access one of those variables,
they can be listed in the capture clause. There are many ways that this can be done.

 \textbf{[]} No variables in the surrounding area are used in the lambda body. This is what we had in previous examples.
 
 \textbf{[varname]} The variable listed will be captured by value. This means that it will make a copy of the value where it can be used later.
 
 \textbf{[\& varname]} The variable listed will be captured by reference. That means that the lambda
 expression can change the value of the variable
 
 \textbf{[\&]} All variables listed in the body are captured by reference. Instead of having an empty [], the \& symbol is put there. This means that 
that if the variable is changed in the lambda expression, it will be changed outside the expression. No variables need to be listed specifically.

One thing that you need to watch out is when you run a lambda that captures a variable by reference and it runs asynchronously. It is very important that the variable that is captured does not leave scope before the lambda is called. If it does, you may get an access violation error.

 \textbf{[=]} All variables listed in the body are captured by value. All the variables that were captured are copied and will work in parallel or asynchronously. It is acceptable for the variables that were captured by value to have left scope before the expression runs.
 
 It is also possible to list variables that will be using a different method of capture. For example: [=, \&varname]. That would capture everything but varname by value. Varname would be by reference. [\&, varname] would capture everything by reference except varname, which 
 would be by value.

\subsubsection{Generalized captures}
C++14 added the capability to create and initialize a variable in the capture section. This
is great for something that you wanted to be able to change, but it would have no longer
been in scope by the time the expression ran. For example:

\begin{lstlisting}
// make a int array of size 7
auto ptr = std::make_unique<int[]>(7); 

auto lambda = [value = std::move(ptr)] {
    /* Something in here that uses "value"*/
    };
\end{lstlisting}

\subsection{Mutable}
If a variable is captured by value, it is treated like a constant. If code in the lambda
body changes the value, the compile flags it as an error. You can put the word "mutable"
in your lambda expression definition. This allows you to change the variables that were 
captured by value. NOTE: It will not be changed outside the lambda, only inside the lambda.

\section{std::function}
\label{stdfunction}
C++ introduced different ways to do function pointers since C++11. In order to use this
technique, the code needs to: 
\begin{lstlisting}
#include <functional>
\end{lstlisting}
The variables that are created this way can store function pointers, lambda expressions, 
function objects, and bind expressions. Once those are stored, these are now callable.

To use the template, it follows this pattern:
\begin{verbatim}
 std::function<retval(params)> fvar;    
\end{verbatim}
The format is a lot easier than a typical function pointer. The
return value (in the pattern it is retval) is the first thing listed.
The items inside the parentheses are the parameter types. The parameters
do not need to be named here, only the types.

Here is an example using a function pointer:
\begin{lstlisting}
// This is a normal function
void print_num(int i) { std::cout << i << '\n'; }

// Here it uses the printnum function, and f_display 
// is set to that function.
  std::function<void(int)> f_display = print_num;
  
// Using the variable, it can now call 
// the print_num function
  f_display(35);
\end{lstlisting}

This tool also works with function objects.
\begin{lstlisting}
// Here is the functor's definition
struct PrintNum {
  void operator()(int i) const { std::cout << i << '\n'; }
};

  // store a functor
  std::function<void(int)> f_functor = PrintNum();
  // call the functor
  f_functor(75);
\end{lstlisting}

And this shows how it can work with a Lambda expression.
\begin{lstlisting}

int main() {
  // store a lambda
  std::function<void()> 
    f_display_89 = []() { print_num(89); };
  f_display_89();

\end{lstlisting}

\subsection{Problems with returned reference}
If you are using std::function that has a returned reference with a lambda expression that does not specify a return value, there can be problems. Quite often it will connect the
return value to a variable that will lose scope as soon as the expression is over. This
will not flag an error with the compiler until C++23.

\subsection{Predefined functions}
Before you create your own lambda expression or functor, you may want to check if what
you were planning on making already exists. There are many simple checks, like less, greater, equal\_to and more defined in the functional header.

Here is an example:
\begin{lstlisting}
template <typename X, typename Y, typename Z = std::less<>>
bool isless(X x, Y y, Z l = Z{})
{
    return l(x, y);
}
 
// and in a function, you can have:
std::cout << std::boolalpha << isless(1, 2);
\end{lstlisting}

\section{Glossary}

\begin{description}
\item[lambda expression:] A technique to create a function inline without naming it.
\item[functor:] A function object. This is a class (or a struct) that has overloaded the parenthesis operator in order to allow it to act like a function.
\item[function pointer:] a variable that is connected to a function instead of data.
\end{description}
