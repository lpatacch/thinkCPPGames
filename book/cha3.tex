% LaTeX source for textbook ``ThinkCPP , a game perspective''
% Copyright (C) 2023  Lisa Patacchiola, Allen B. Downey


\chapter{Conditionals}
\label{conditional}

\section{Floating-point}
\index{floating-point number}
\index{type!double}
\index{double (floating-point)}
\label{floating-point}

In the last chapter we had some problems dealing with numbers
that were not integers.  We worked around the problem by measuring
percentages instead of fractions, but a more general solution is
to use floating-point numbers, which can represent fractions
as well as integers.  In C++, there are two floating-point types,
called {\tt float} and {\tt double}.  In this book we will use
{\tt double}s exclusively.

You can create floating-point variables and assign values to them
using the same syntax we used for the other types.  For example:

\begin{lstlisting}
  double pi;
  pi = 3.14159;
\end{lstlisting}
%
It is also legal to declare a variable and assign a value to it at the
same time:

\begin{lstlisting}
  int x = 1;
  string empty = "";
  double pi = 3.14159;
\end{lstlisting}
%
In fact, this syntax is quite common.  A combined declaration
and assignment is sometimes called an {\bf initialization}.

\index{initialization}

Although floating-point numbers are useful, they are
often a source of confusion because there seems to be an
overlap between integers and floating-point numbers.  For
example, if you have the value {\tt 1}, is that an integer,
a floating-point number, or both?

Strictly speaking, C++ distinguishes the integer value {\tt 1}
from the floating-point value {\tt 1.0}, even though they
seem to be the same number.  They belong to
different types, and strictly speaking, you are not allowed
to make assignments between types.  For example, the following
is illegal

\begin{lstlisting}
    int x = 1.1;
\end{lstlisting}
%
because the variable on the left is an {\tt int}
and the value on the right is a {\tt double}.  But it is easy
to forget this rule, especially because there are places where C++
automatically converts from one type to another.
For example,

\begin{lstlisting}
    double y = 1;
\end{lstlisting}
%
should technically not be legal, but C++ allows it by converting the
{\tt int} to a {\tt double} automatically.  This leniency is
convenient, but it can cause problems; for example:

\begin{lstlisting}
    double y = 1 / 3;
\end{lstlisting}
%
You might expect the variable {\tt y} to be given the value
{\tt 0.333333}, which is a legal floating-point value, but in
fact it will get the value {\tt 0.0}.  The reason is that the
expression on the right appears to be the ratio of two integers,
so C++ does {\em integer} division, which yields the integer
value {\tt 0}.  Converted to floating-point, the result is
{\tt 0.0}.

One way to solve this problem (once you figure out what
it is) is to make the right-hand side a floating-point
expression:

\begin{lstlisting}
    double y = 1.0 / 3.0;
\end{lstlisting}
%
This sets {\tt y} to {\tt 0.333333}, as expected.

\index{arithmetic!floating-point}

All the operations we have seen---addition, subtraction,
multiplication, and division---work on floating-point values,
although you might be interested to know that the underlying mechanism
is completely different.  In fact, most processors have special
hardware just for performing floating-point operations.

\subsection{Converting from {\tt double} to {\tt int}}
\label{rounding}
\index{rounding}
\index{typecasting}

As I mentioned, C++ converts {\tt int}s
to {\tt double}s automatically if necessary, because no
information is lost in the translation.  On the other hand,
going from a {\tt double} to an {\tt int} requires rounding
off.  C++ doesn't perform this operation automatically, in
order to make sure that you, as the programmer, are aware
of the loss of the fractional part of the number.

The simplest way to convert a floating-point value to an integer is to
use a {\bf typecast}.  Typecasting is so called because it allows you
to take a value that belongs to one type and ``cast'' it into another
type (in the sense of molding or reforming, not throwing).

The syntax for typecasting is like the syntax
for a function call.  For example:

\begin{verbatim}
  double pi = 3.14159;
  int x = int (pi);
\end{verbatim}
%
The {\tt int} function returns an integer, so {\tt x} gets the value
3.  Converting to an integer always rounds down, even if the fraction
part is 0.99999999.

FIXME ADD STATIC CAST HERE

For every type in C++, there is a corresponding function that
typecasts its argument to the appropriate type.

\section{Conditional execution}
\index{conditional}
\index{statement!conditional}

In order to write useful programs, we almost always need the ability
to check certain conditions and change the behavior of the program
accordingly.  {\bf Conditional statements} give us this ability.  The
simplest form is the {\tt if} statement:

\begin{lstlisting}
  if (x > 0) {
    std::cout << "x is positive" << std::endl;
  }
\end{lstlisting}
%
The expression in parentheses is called the condition.
If it is true, then the statements in brackets get executed.
If the condition is not true, nothing happens.

\index{operator!comparison}
\index{comparison!operator}

The condition can contain any of the {\tt comparison operators}:

\begin{lstlisting}
    x == y               // x equals y
    x != y               // x is not equal to y
    x > y                // x is greater than y
    x < y                // x is less than y
    x >= y               // x is greater than or equal to y
    x <= y               // x is less than or equal to y
\end{lstlisting}
%
Although these operations are probably familiar to you, the
syntax C++ uses is a little different from mathematical
symbols like $=$, $\neq$ and $\le$.  A common error is
to use a single {\tt =} instead of a double {\tt ==}.  Remember
that {\tt =} is the assignment operator, and {\tt ==} is
a comparison operator.  Also, there is no such thing as
{\tt =<} or {\tt =>}.

The two sides of a condition operator have to be the same
type.  You can only compare {\tt ints} to {\tt ints} and
{\tt doubles} to {\tt doubles}.  Unfortunately, at this
point you can't compare {\tt string}s at all!  There is
a way to compare {\tt string}s, but we won't get to it for a couple
of chapters.

\section {Alternative execution}
\label{alternative}
\index{conditional!alternative}

A second form of conditional execution is alternative execution,
in which there are two possibilities, and the condition determines
which one gets executed.  The syntax looks like:

\begin{lstlisting}
    
  if (x%2 == 0) {
    cout << "x is even" << endl;
  } else {
    cout << "x is odd" << endl;
  }
\end{lstlisting}
%
(If you do not remember what the \% sign is doing, check out the
Modulus section in Chapter~\ref{modulus})
If the remainder when {\tt x} is divided by 2 is zero, then
we know that {\tt x} is even, and this code displays a message
to that effect.  If the condition is false, the second
set of statements is executed.  Since the condition must
be true or false, exactly one of the alternatives will be
executed.

As an aside, if you think you might want to check the parity
(evenness or oddness) of numbers often, you might want to
``wrap'' this code up in a function, as follows:

\begin{verbatim}
void printParity (int x) {
  if (x%2 == 0) {
    cout << "x is even" << endl;
  } else {
    cout << "x is odd" << endl;
  }
}
\end{verbatim}
%
Now you have a function named {\tt printParity} that will display
an appropriate message for any integer you care to provide.
In {\tt main} you would call this function as follows:

\begin{verbatim}
    printParity (17);
\end{verbatim}
%
Always remember that when you {\em call} a function, you do
not have to declare the types of the arguments you provide.
C++ can figure out what type they are.  You should resist the
temptation to write things like:

\begin{verbatim}
  int number = 17;
  printParity (int number);         // WRONG!!!
\end{verbatim}

\section {Chained conditionals}
\index{conditional!chained}

Sometimes you want to check for a number of related conditions
and choose one of several actions.  One way to do this is by
{\bf chaining} a series of {\tt if}s and {\tt else}s:

\begin{verbatim}
  if (x > 0) {
    cout << "x is positive" << endl;
  } else if (x < 0) {
    cout << "x is negative" << endl;
  } else {
    cout << "x is zero" << endl;
  }
\end{verbatim}
%
These chains can be as long as you want, although they can
be difficult to read if they get out of hand.  One way to
make them easier to read is to use standard indentation,
as demonstrated in these examples.  If you keep all the
statements and squiggly-braces lined up, you are less
likely to make syntax errors and you can find them more
quickly if you do.

\section{Nested conditionals}
\index{conditional!nested}

In addition to chaining, you can also nest one conditional
within another.  We could have written the previous example
as:

\begin{verbatim}
  if (x == 0) {
    cout << "x is zero" << endl;
  } else {
    if (x > 0) {
      cout << "x is positive" << endl;
    } else {
      cout << "x is negative" << endl;
    }
  }
\end{verbatim}
%
There is now an outer conditional that contains two branches.  The
first branch contains a simple output statement, but the second
branch contains another {\tt if} statement, which has two branches
of its own.  Fortunately, those two branches are both output
statements, although they could have been conditional statements as
well.

Notice again that indentation helps make the structure
apparent, but nevertheless, nested conditionals get difficult to read
very quickly.  In general, it is a good idea to avoid them when you
can.

\index{nested structure}

On the other hand, this kind of {\bf nested structure} is common, and
we will see it again, so you better get used to it.

\section{Pseudocode}
\section{Flowcharts}
\section{Glossary}

\begin{description}
\item[floating-point:] A type of variable (or value) that can contain
fractions as well as integers.  There are a few floating-point types
in C++; the one we use in this book is {\tt double}.

\item[conditional:]  A block of statements that may or may not
be executed depending on some condition.

\item[chaining:]  A way of joining several conditional statements
in sequence.

\item[nesting:] Putting a conditional statement inside one or both
branches of another conditional statement.



\index{conditional}
\index{conditional!chained}
\index{conditional!nested}
\index{floating-point}

\end{description}


