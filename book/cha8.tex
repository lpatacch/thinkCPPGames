% LaTeX source for textbook ``How to think like a computer scientist''
% Copyright (C) 2023  Lisa Patacchiola and Allen B. Downey


\chapter{Structures}
\label{structs}
\index{struct}

\section{Compound values}

Most of the data types we have been working with represent a single
value---an integer, a floating-point number, a boolean value.  {\tt
string}s are different in the sense that they are made up of smaller
pieces, the characters.  Thus, {\tt string}s are an example of a
{\bf compound} type.

Depending on what we are doing, we may want to treat a compound type
as a single thing (or object), or we may want to access its parts (or
instance variables).  This ambiguity is useful.

It is also useful to be able to create your own compound values.  C++
provides two mechanisms for doing that: {\bf structures} and {\bf
classes}.  We will start out with structures and get to classes in
Chapter~\ref{class} \footnote{If you are coming from C, structures have more features in C++. C does not have member functions in its structures.}

\section{{\tt Point} objects}
\index{Point}
\index{struct!Point}

As a simple example of a compound structure, consider the concept of a
mathematical point.  At one level, a point is two numbers
(coordinates) that we treat collectively as a single object.  In
mathematical notation, points are often written in parentheses, with a
comma separating the coordinates.  For example, $(0, 0)$ indicates the
origin, and $(x, y)$ indicates the point $x$ units to the right and
$y$ units up from the origin.

A natural way to represent a point in C++ is with two {\tt double}s.
The question, then, is how to group these two values into
a compound object, or structure.  The answer is a {\tt struct}
definition:

\begin{verbatim}
struct Point {
  double x, y;
};  
\end{verbatim}
%
{\tt struct} definitions appear outside of any function definition,
usually at the beginning of the program (after the {\tt include}
statements).

This definition indicates that there are two elements in this
structure, named {\tt x} and {\tt y}.  These elements are called
{\bf instance variables}, for reasons I will explain a little
later.

It is a common error to leave off the semi-colon at the end of a
structure definition.  It might seem odd to put a semi-colon after a
curly-brace, but you'll get used to it.

Once you have defined the new structure, you can create variables
with that type:

\begin{verbatim}
  Point blank;
  blank.x = 3.0;
  blank.y = 4.0;   
\end{verbatim}
%
The first line is a conventional variable declaration: {\tt blank} has
type {\tt Point}.  The next two lines initialize the instance variables of the
structure.  The ``dot notation'' used here is similar to the syntax
for invoking a function on an object, as in {\tt fruit.length()}.
Of course, one difference is that function names are always followed
by an argument list, even if it is empty.

\index{declaration}
\index{statement!declaration}
\index{reference}
\index{state diagram}
\index{state}

The result of these assignments is shown in the following
state diagram:

\vspace{0.1in}
\centerline{\epsfig{figure=oldimages/point.eps}}
\vspace{0.1in}

As usual, the name of the variable {\tt blank} appears outside the box
and its value appears inside the box.  In this case, that value is
a compound object with two named instance variables.

\section{Accessing instance variables}
\index{struct!instance variable}

You can read the values of an instance variable using the same syntax we
used to write them:

\begin{verbatim}
    int x = blank.x;
\end{verbatim}
%
The expression {\tt blank.x} means ``go to the object named {\tt
blank} and get the value of {\tt x}.''  In this case we assign that
value to a local variable named {\tt x}.  Notice that there is no
conflict between the local variable named {\tt x} and the instance
variable named {\tt x}.  The purpose of dot notation is to identify
{\em which} variable you are referring to unambiguously.

You can use dot notation as part of any C++ expression, so the
following are legal.

\begin{verbatim}
  std::cout << blank.x << ", " << blank.y << std::endl;
  double distance = blank.x * blank.x + blank.y * blank.y;
\end{verbatim}
%
The first line outputs {\tt 3, 4}; the second line calculates
the value 25.

\section{Operations on structures}
\index{struct!operations}

Most of the operators we have been using on other types, like
mathematical operators ( {\tt +}, {\tt \%}, etc.) and comparison
operators ({\tt ==}, {\tt >}, etc.), do not work on structures.
Actually, it is possible to define the meaning of these operators
for the new type, but we won't do that until Section \ref{overloadingops}.

On the other hand, the assignment operator {\em does} work for
structures.  It can be used in two ways: to initialize the instance
variables of a structure or to copy the instance variables from one
structure to another.  An initialization looks like this:

\begin{verbatim}
  Point blank = { 3.0, 4.0 };
\end{verbatim}
%
The values in curly braces get assigned to the instance variables of
the structure one by one, in order.  So in this case, {\tt x}
gets the first value and {\tt y} gets the second.

Unfortunately, this syntax can be used only in an initialization,
not in an assignment statement.  So the following is illegal.

\begin{verbatim}
  Point blank;
  blank = { 3.0, 4.0 };       // WRONG !!
\end{verbatim}
%
You might wonder why this perfectly reasonable statement should
be illegal; I'm not sure, but I think the problem is that the compiler
doesn't know what type the right hand side should be.  If you
add a typecast:

\begin{verbatim}
  Point blank;
  blank = (Point){ 3.0, 4.0 };
\end{verbatim}
%
That works.

It is legal to assign one structure to
another.  For example:

\begin{verbatim}
  Point p1 = { 3.0, 4.0 };
  Point p2 = p1;
  cout << p2.x << ", " <<  p2.y << endl;
\end{verbatim}
%
The output of this program is {\tt 3, 4}.

\section{Structures as parameters}
\index{parameter}
\index{struct!as parameter}

You can pass structures as parameters in the usual way.  For
example,

\begin{verbatim}
void printPoint (Point p) {
  std::cout << "(" << p.x << ", " << p.y << ")" << std::endl;
}
\end{verbatim}
%
{\tt printPoint} takes a point as an argument and outputs it in
the standard format.  If you call {\tt printPoint (blank)},
it will output {\tt (3, 4)}.

As a second example, we can rewrite the {\tt distance} function from
Section~\ref{distance} so that it takes two {\tt Point}s as parameters
instead of four {\tt double}s.

\begin{verbatim}
double distance (Point p1, Point p2) {
  double dx = p2.x - p1.x;
  double dy = p2.y - p1.y;
  return sqrt (dx*dx + dy*dy);
}
\end{verbatim}

\section{Call by value}
\index{parameter passing}
\index{call by value}

When you pass a structure as an argument, remember that the
argument and the parameter are not the same variable.  Instead,
there are two variables (one in the caller and one in the
callee) that have the same value, at least initially.  For
example, when we call {\tt printPoint}, the stack diagram
looks like this:

\vspace{0.1in}
\centerline{\epsfig{figure=oldimages/point2.eps}}
\vspace{0.1in}
%
If {\tt printPoint} happened to change one of the instance variables
of {\tt p}, it would have no effect on {\tt blank}.  Of course, there
is no reason for {\tt printPoint} to modify its parameter, so this
isolation between the two functions is appropriate.

This kind of parameter-passing is called ``pass by value''
because it is the value of the structure (or other type) that
gets passed to the function.

\section{Call by reference}
\index{parameter passing}
\index{call by reference}
\index{reference}
\label{reference}

An alternative parameter-passing mechanism that is available
in C++ is called ``pass by reference.''  This mechanism makes
it possible to pass a structure to a procedure and modify it.

For example, you can reflect a point around the 45-degree line by
swapping the two coordinates.  The most obvious (but incorrect) way to
write a {\tt reflect} function is something like this:

\begin{verbatim}
void reflect (Point p)      // WRONG !!
{
  double temp = p.x;
  p.x = p.y;
  p.y = temp;
}
\end{verbatim}
%
But this won't work, because the changes we make in {\tt reflect}
will have no effect on the caller.

Instead, we have to specify that we want to pass the parameter by
reference.  We do that by adding an ampersand ({\tt \&}) to the
parameter declaration:

\begin{verbatim}
void reflect (Point& p)
{
  double temp = p.x;
  p.x = p.y;
  p.y = temp;
}
\end{verbatim}
%
Now we can call the function in the usual way:

\begin{verbatim}
 Point p1 = { 3.0, 4.0 };
 
  printPoint (p1);
  reflect (p1);
  printPoint (p1);
\end{verbatim}
%
The output of this program is as expected:

\begin{verbatim}
(3, 4)
(4, 3)
\end{verbatim}
%
Here's how we would draw a stack diagram for this program:

\vspace{0.1in}
\centerline{\epsfig{figure=oldimages/point3.eps}}
\vspace{0.1in}
%
The parameter {\tt p} is a reference to the structure named {\tt
blank}.  The usual representation for a reference is a dot with an
arrow that points to whatever the reference refers to.

The important thing to see in this diagram is that any changes that
{\tt reflect} makes in {\tt p} will also affect {\tt blank}.

Passing structures by reference is more versatile than passing by
value, because the callee can modify the structure.  It is also
faster, because the system does not have to copy the whole
structure.  On the other hand, it is less safe, since it is harder to
keep track of what gets modified where.  Nevertheless, in C++
programs, almost all structures are passed by reference almost all the
time.  In this book I will follow that convention.

If you do not expect to change the object in the function, another method is to use constant references. This makes using references safer:
\begin{verbatim}
void printPoint (const Point &p) {
  std::cout << "(" << p.x << ", " << p.y << ")" << std::endl;
}
\end{verbatim}
The above code passes the structure by reference, but the function is not allowed to change it. 

\section{Rectangles}
\index{Rectangle}
\index{struct!Rectangle}

Now let's say that we want to create a structure to represent
a rectangle.  The question is, what information do I have to
provide in order to specify a rectangle?  To keep things simple
let's assume that the rectangle will be oriented vertically or
horizontally, never at an angle.

There are a few possibilities: I could specify the center of
the rectangle (two coordinates) and its size (width and height),
or I could specify one of the corners and the size, or I
could specify two opposing corners.

The most common choice in existing programs is to specify the
upper left corner of the rectangle and the size.  To do that
in C++, we will define a structure that contains a {\tt Point}
and two doubles.

\begin{verbatim}
struct Rectangle {
  Point corner;
  double width, height;
};  
\end{verbatim}
%
Notice that one structure can contain another.  In fact, this
sort of thing is quite common.  Of course, this means that in
order to create a {\tt Rectangle}, we have to create a {\tt Point}
first:

\begin{verbatim}
  Point corner = { 0.0, 0.0 };
  Rectangle box = { corner, 100.0, 200.0 };
\end{verbatim}
%
This code creates a new {\tt Rectangle} structure and initializes the
instance variables.  The figure shows the effect of this assignment.

\vspace{0.1in}
\centerline{\epsfig{figure=oldimages/rectangle.eps}}
\vspace{0.1in}
%
We can access the {\tt width} and {\tt height} in the usual way:

\begin{verbatim}
  box.width += 50.0;
  std::cout << box.height << std::endl;
\end{verbatim}
%
In order to access the instance variables of {\tt corner}, we can use a
temporary variable:

\begin{verbatim}
  Point temp = box.corner;
  double x = temp.x;
\end{verbatim}
%
Alternatively, we can compose the two statements:

\index{composition}

\begin{verbatim}
  double x = box.corner.x;
\end{verbatim}
%
It makes the most sense to read this statement from right to
left: ``Extract {\tt x} from the {\tt corner} of the {\tt box},
and assign it to the local variable {\tt x}.''

While we are on the subject of composition, I should point
out that you can, in fact, create the {\tt Point} and the
{\tt Rectangle} at the same time:

\begin{verbatim}
  Rectangle box = { { 0.0, 0.0 }, 100.0, 200.0 };
\end{verbatim}
%
The innermost curly braces are the coordinates of the
corner point; together they make up the first of the three
values that go into the new {\tt Rectangle}.  This statement
is an example of {\bf nested structure}.

\index{nested structure}


\section{Structures as return types}
\index{struct!as return type}
\index{return}
\index{statement!return}

You can write functions that return structures.  For example,
{\tt findCenter} takes a {\tt Rectangle} as an argument and
returns a {\tt Point} that contains the coordinates of the
center of the {\tt Rectangle}:

\begin{verbatim}
Point findCenter (const Rectangle& box)
{
  double x = box.corner.x + box.width/2;
  double y = box.corner.y + box.height/2;
  Point result = {x, y};
  return result;
}
\end{verbatim}
%
To call this function, we have to pass a box as an argument
(notice that it is being passed by reference), and assign the
return value to a {\tt Point} variable:

\begin{verbatim}
  Rectangle box = { {0.0, 0.0}, 100, 200 };
  Point center = findCenter (box);
  printPoint (center);
\end{verbatim}
%
The output of this program is {\tt (50, 100)}.

\section {Passing other types by reference}
\index{parameter passing}
\index{call by reference}
\index{reference}

It's not just structures that can be passed by reference.
All the other types we've seen can, too.  For example, to swap
two integers, we could write something like:

\begin{verbatim}
void swap (int& x, int& y)
{
  int temp = x;
  x = y;
  y = temp;
}
\end{verbatim}
%
We would call this function in the usual way:

\begin{verbatim}
  int i = 7;
  int j = 9;
  swap (i, j);
  std::cout << i << j << std::endl;
\end{verbatim}
%
The output of this program is {\tt 97}.  Draw a stack
diagram for this program to convince yourself this is true.
If the parameters {\tt x} and {\tt y} were declared as
regular parameters (without the {\tt \&}s), {\tt swap} would
not work.  It would modify {\tt x} and {\tt y} and have no
effect on {\tt i} and {\tt j}.

When people start passing things like integers by reference,
they often try to use an expression
as a reference argument.  For example:

\begin{verbatim}
  int i = 7;
  int j = 9;
  swap (i, j+1);         // WRONG!!
\end{verbatim}
%
This is not legal because the expression {\tt j+1} is not
a variable---it does not occupy a location that the reference
can refer to.  It is a little tricky to figure out exactly
what kinds of expressions can be passed by reference.  For now,
a good rule of thumb is that non-const reference arguments have to be
variables.

\section{Glossary}

\begin{description}

\item[structure:]  A collection of data and methods grouped together and
treated as a single object.

\item[instance variable:]  One of the named pieces of data that make up
a structure.

\item[reference:]  A value that indicates or refers to a variable
or structure.  In a state diagram, a reference appears as an arrow.

\item[pass by value:]  A method of parameter-passing in which the
value provided as an argument is copied into the corresponding
parameter, but the parameter and the argument occupy distinct
locations.

\item[pass by reference:]  A method of parameter-passing in which
the parameter is a reference to the argument variable.  Changes
to the parameter also affect the argument variable.

\index{structure}
\index{instance variable}
\index{reference}
\index{pass by value}
\index{pass by reference}

\end{description}

